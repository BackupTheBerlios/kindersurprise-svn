\documentclass{article}

\begin{document}

\tableofcontents

\newpage

\section*{KinderSurprise}

Date of Create: 28.01.2011\\
Date of last Update: 08.02.2011\\
Author: Enrico Gallus\\

Supposed to be a web-based c\# application used to save and organize a
kindersurprise collection. 
\subsection{Aim}
\begin{itemize}
  \item web based application
  \item using database to save information
  \item platform independent
  \item user-specific acces to the data
  \item provide a simple system to show information of categories, series and
  figures
  \item make every item editable, deletable and addable
  \item every serie and figure should consist of a store
  \item support images for series and figures
  \item import/export functionality for all the data
  \item has ebay-interface to sell figures/series
  \item offers all functions of a customer-based trading center
\end{itemize}

\subsection{Developing Tools}
I used to work under Debian Testing (Squeeze), using the following Repository
for Mono: ``deb http://debian.meebey.net/pkg-mono /''

\begin{itemize}
  \item MonoDevelop v$2.4$
  \item Mono-Framework
  \item NHibernate v$2.1.2.4000$
  \item FluentNHibernate v$1.1$
  \item MySql v$5.1$
  \item NUnit v$2.4.7$
  \item Moq $v4.0.10827$
  \item NhLambdaExtensions $v1.0.10.0$
  \item structuremap $v2.6.1$
  \item rabbitvcs-nautilus
\end{itemize}

\subsection{Patterns}
\begin{itemize}
  \item Model-View-Presenter
  \item Dependency Injection
  \item N-Tier
  \item Test-Driven-Development
  \item UnitOfWork
\end{itemize}

\subsection{Installation}

\subsubsection{Mono}
To use the C\# programming language and asp.net under linux we need to install
a special framework. This framework is called mono.
You can install it using the following command with root rights:
\begin{verbatim}
apt-get install mono-complete
\end{verbatim}
An alternative under windows to visual studio is MonoDevelop. MonoDevelop is a
Integrated Development Enviroment especially for C\#.
Installation can be done with:
\begin{verbatim}
apt-get install monodevelop
\end{verbatim}

\subsubsection{VCS}
Get the sourcecode of the project via Berlios.
I recommend the rabbitvcs-nautilus package from the squeeze distribution.
Handling vcs can't be easier and you can handle everything with nautilus.
To install please type in
\begin{verbatim}
apt-get install rabbitvcs-nautilus
\end{verbatim}
You need root rights to execute this command.

\subsubsection{Setup DBMS}
The KinderSurprise application is using the mysql database managament system.
There is also a possibility to use the application with any other dbms, but to
keep the preparation as easy as possible, i will explain how to setup
mysql-server and the client.
Open a command line and get root rights.
Type in the following command 
\begin{verbatim}
apt-get install mysql-server mysql-client
\end{verbatim} 
During the install process mysql will ask you to setup a root password.
Write down this password, so you can never ever loose it.
If the installation is finished you can use the command mysql to login to any
database.
I use phpmyadmin, a web gui for mysql. Installation is pretty easy:
\begin{verbatim}
apt-get install phpmyadmin
\end{verbatim}


\subsubsection{Setup Database}
The first step is to set up the database and testdatabase with MySql.
Please create an user for your mysql instance and create a database named
``kindersurprise'' and ``testkindersurprise''.
Afterwards open the projectdir and change to the script folder.
In there you will find all scripts necessary for the database structure. The
name is organised as date followed by the function.
Please execute the containing scripts for both databases in chronological order.
If no errors occured and the structure was created on both databases, than
please also execute the testscripts in the test database.

\subsubsection{Setup Database Connection}
Please open the project with MonoDevelop. Use the sln-file in the projectDir to
open the solution.
You have to edit the WebTest.config file in the following projects
\begin{itemize}
  \item KinderSurprise.DAL.Test
  \item KinderSurprise.MVP.Model.Test
  \item KinderSurprise.MVP.Presenter.Test
\end{itemize}
Customize the KinderSurpriseConnection String. Set the database name to
``testkindersurprise'' if you used the name for the testdatabase. Also change
the ID to the user you have created in the step before. Change the password.\\

Also edit the Web.Config in the KinderSurprise.MVP.View subproject and make the
same changes but instead of using the testdatabase use ``kindersurprise''.

\subsubsection{Last Step}
To verify if everything works well, please execute the unittests defined in the
the three test projects.
Compile the project and try to start.
If any problem or error occure, please don't hesitate to contact me.

\subsection{Using .net under Visual Studio}
If you rather would use visual studio under windows than to use the mono
framework, you have to change to modify some parts of the project.
After open the solution file of the project the ui-project will maybe not be
loaded correctly. If this error occured than you have to modify the solution
file of the $KinderSurprise.MVP.View$ project like described in the file
$windows_changes.txt$.
Please comment out the part existing in the $windows_changes.txt$ in the
solution file of the $KinderSurprise.MVP.View$. And copy the
$windows_changes.txt$ content in the soltution file.
Afterwards reload the project.
\newline
On the other side the $Nunit.framework.dll$ does not belong to the lib order.
Mono is using the class directly from the system. So the references under visual
studio are not correct. Delete all incorrect references to the dll in all the
test projects. Copy the Nunit.Framework.dll to the lib directory and
reference in the test-projects this libary.

\subsection{Workflows}
\subsubsection{Working with the database}
If you make changes to the database, please save all the commands to a script
which can be added to the script-folder. So other developer has it as easy as
you, to setup the project. No need to mention the filename-specification!?
\subsubsection{Submitting of Code changes}
If you want to commit any changes. Please be sure there are no broken unittests.
I'm still looking for a codecoverage tool under linux which works with c\#. If
you have any proposal, so please let me know. If there are more developer, maybe
we can think about using continous integration.

\subsection{Workaround and Structure}
\subsubsection{The workaround}
The user interact with our application.
The view call the presenter and will tell the current state of the user control
elements and so on. The presenter delegates the task to the model for the
business logic and the DAL to collect and edit the information from the
database. 
The dal is an interface for the database. The model is the workaround for our application.
The Bootstrap layer is the implementation of the dependency injection pattern
and the NHBootstrap layer represents the unit of work pattern for NHibernate.
The Test-Layers are the programmatically documentation.
DAL.Fake will be used to fake the database data and make unittests independend
of the database.

\subsubsection{View -The GUI}
The view is the layer which provides the graphical user interface
implemented with asp.net framework.
The layer currently consists of one default page and an usercontrol for
category, figur and serie.
There exists also an account part, but it was automatically generated and should
be reworked in the future.
The default page contains a treeview on the left site to organize a view of all
categories, series and figurs in a relation.
If an element in the treeview was clicked the corresponding usercontrol should
be open on the default page.
Here you can work with the items, e.g. edit the properties. On the other site
you sould be able to delete, add and move the item to another parent item.
\newline
Each page and usercontrol contains all properties (user control elements) as
encapsulation. There is no business logic in this layer. All handling should be
done by the presenter.
\newline
Featues:
\begin{itemize}
  \item Think about an icon for the application
  \item The account
  \begin{itemize}
    \item Register and Login should be implemented in c\# and using less
    javascript (none if possible)
    \item Think of a secure way to implement the login handling
    (password-encrypting-decrypting, ..)
    \item Different Options should be activated if the user is logged in.
    \begin{itemize}
      \item layout for the account, e.g. change personal information or
      password
      \item The trading function should be activated
      \item own message system to let user communicate with each other
    \end{itemize}
  \end{itemize}
  \item User notification if he wants to delete an item which has still childs.
\end{itemize}

\subsubsection{The presenter}
The interfaces defined in the presenter layer contain all the user controls of
our gui. In this layer we can handle the changes of the layout of the user
controls and use the model for the business behaviour.

\subsubsection{The model}
The model contains service methods for the repositories. And business code like
the picturehandling, an validator and so on.

\subsubsection{DAL - data access layer}
The data access layer got on class for each table in the database.
This is the lowest layer we can reach. NHibernate is used to handle the object
- relational mapping between our application and the database in the background.
The Code is in such a way encapsulated that changing the dbms does not
require code changes.
The repositories provide one method for one process (save, update, delete,
getting by id or more complex queries.

\subsubsection{Database model}
The Layer Model contains the objects we are using in the application.
This objects are used by the mapper.

\subsubsection{Mapper}
The mapper is using the FluentNHibernate framework to ``connect'' our objects
with the corresponding relational tables in the database.

\subsubsection{Bootstrap}
Bootstrap contains our solution for the dependency injection pattern.
There are a productive and a testing container.

\subsubsection{NHBootstrap}
This layer contains our implementation of the unit of work pattern.

\subsubsection{Test}
Some projects have an corresponding test project.
Here is the testing of the behaviour of our units.
\end{document}
